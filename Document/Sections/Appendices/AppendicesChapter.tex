\section{Appendices}

\subsection{Pressure Measurement}

Assumptions for pressure measurement calculations in Urodynamics Without Borders system.

Gravity has been assumed to be:
\gravityconstant

Water density has been assumed to be:
\waterdensityconstant


The pressure exerted by a static fluid column is given by the equation, the gravity and water density constants are used to express pressure in \cmhtwosymbol
\pressurefluidequation

The pressures used in all calculations are the gauge pressures (relative to atmospheric pressure) not the absolute pressures. 


\chapterauthorsbox{\authmj and \authem}



\subsection{Uroflowmetry Measurement}

\subsubsection{Hydrostatic Pressure Relation}
\label{appendix:hydrostatic}

For these tests, assume that 1 cm $\approx$ 1 cmH\textsubscript{2}O using:
\[
    P = \rho g h_w
\]
where  
\begin{tabular}{ll}
$P$   & hydrostatic pressure (cmH\textsubscript{2}O) \\
$\rho$ & fluid density ($\approx 1000$ kg/m\textsuperscript{3}) \\
$g$   & 9.81 m/s\textsuperscript{2} \\
$h_w$ & height difference (m) \\
\end{tabular}

\subsubsection{Torricelli’s Law}
\label{appendix:torricelli}

Torricelli’s law describes jet speed based on fluid height:
\[
    v = C_d \sqrt{2 g h_w}
\]
where  
$v$ = outlet velocity,  
$C_d$ = discharge coefficient (0.6--0.8).

Flow rate:
\[
    Q = A v = C_d A \sqrt{2 g h_w}.
\]

Solving for $h_w$:
\[
    h_w = \frac{1}{2g} \left( \frac{Q_t}{C_d A} \right)^2.
\]

Example (50 mL/s, 5 mm tube, $C_d = 0.7$):
\[
A = \pi\left(\frac{0.005}{2}\right)^2 = 1.96 \times 10^{-5} \text{ m}^2,
\]
\[
h_w = 0.68 \text{ m}.
\]

\begin{table}[H]
    \centering
    \begin{tabular}{|c|c|c|}
        \hline
        \textbf{Flow rate (mL/s)} & \textbf{$P_w$ (mm)} & \textbf{$h_w$ (mm)} \\ \hline
        10 & 4 & 66  \\ \hline
        20 & 4 & 264 \\ \hline
        30 & 5 & 243 \\ \hline
        40 & 5 & 432 \\ \hline
        50 & 6 & 325 \\ \hline
    \end{tabular}
    \caption{Example values from Torricelli’s law}
\end{table}