\section{Introduction}

A urodynamic (UDS) test evaluates the bladder function by assessing the pressure-volume relationship during the storage phase, and the pressure-flow relationship during the voiding phase. Bladder dysfunction can result in lower urinary tract symptoms such as frequent urination, incontinence and difficulty emptying the bladder. In severe cases, bladder dysfunction results in kidney damage and may be fatal. Determining the root cause of the bladder dysfunction allows clinicians to most effectively treatment and preserve kidney function.  

During a UDS test, fluid is infused to the bladder at a set flowrate. The detrusor pressure (bladder pressure - abdominal pressure) change with infused volume is recorded. Once the patient reaches the functional bladder capacity or incontinence is elicited, permission to void is given and the pressure-flow relationship recorded. 
The standard urodynamic system costs approximately (US) \$30,000. The acquisition and running cost of urodynamics systems are often considered prohibitive for most hospitals in the developing world.


\subsection{Aims}
The aim of this project was to develop and validate a urodynamic system for a total cost of \$100 (excluding a computer), meeting the performance standards set out in the international continence society urodynamic equipment guidelines\ (click \icsguidelines \ to download the ICS guidelines file).

\clearpage